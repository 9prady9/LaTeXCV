
% Pradeep Garigipati's Resume
% Created: 14 Nov 2011
% Last Modified: 14 Nov 2011

\documentclass[11pt,oneside]{article}
\usepackage{geometry}
\usepackage{hyperref}
\usepackage[T1]{fontenc}
\pagestyle{empty}
\geometry{letterpaper,tmargin=0.55in,bmargin=0.55in,lmargin=0.65in,rmargin=0.65in,headheight=0in,headsep=0in,footskip=.3in}
\setlength{\parindent}{0in}
\setlength{\parskip}{0in}
\setlength{\itemsep}{0in}
\setlength{\topsep}{0in}
\setlength{\tabcolsep}{0in}

\hypersetup{
  colorlinks=true,       % false: boxed links; true: colored links
  urlcolor=blue           % color of external links
}

% Name and contact information
\newcommand{\name}{Pradeep Garigipati}
\newcommand{\web}{\href{https://www.linkedin.com/in/pradeepgarigipati}{LinkedIn} , \href{https://github.com/9prady9}{GitHub}}

%%%%%%%%%%%%%%%%%%%%%%%%%%%%%%%%%%%%%%%%%%%%%%%%%%%%%%%%%
% New commands and environments

% This defines how the name looks
\newcommand{\bigname}[1]{
  \begin{center}\fontfamily{phv}\selectfont\Huge\scshape#1\end{center}
}

% A ressection is a main section (<H1>Section</H1>)
\newenvironment{ressection}[1]{
  \vspace{4pt}
  {\fontfamily{phv}\selectfont\Large#1}
  \begin{itemize}
  \vspace{3pt}
}{
  \end{itemize}
}

\newenvironment{ressectionlist}[1]{
  \vspace{4pt}
  {\fontfamily{phv}\selectfont\Large#1}
  \vspace{4pt}
}

% A resitem is a simple list element in a ressection (first level)
\newcommand{\resitem}[1]{
  \vspace{-4pt}
  \item \begin{flushleft} #1 \end{flushleft}
}

% A ressubitem is a simple list element in anything but a ressection (second level)
\newcommand{\ressubitem}[1]{
  \vspace{-1pt}
  \item [$\square$] \begin{flushleft} #1 \end{flushleft}
}

% A resbigitem is a complex list element for stuff like jobs and education:
%  Arg 1: Name of company or university
%  Arg 2: Location
%  Arg 3: Title and/or date range
\newcommand{\resbigitem}[3]{
  \vspace{-5pt}
  \item
  \textbf{#1}, \textnormal #2 \hfill #3
}

% This is a list that comes with a resbigitem
% but has normal text instead of sublist
\newenvironment{ressubsec3}[3]{
  \resbigitem{#1}{#2}{#3}
}

% This is a simple sublist
\newenvironment{reslist}[1]{
  \resitem{\textbf{#1}}
  \vspace{-5pt}
  \begin{itemize}
}{
  \end{itemize}
}

\newenvironment{blockquote}{%
  \par%
  \medskip
  \leftskip=2em\rightskip=2em%
  \noindent\ignorespaces}{%
  \par\medskip}

%%%%%%%%%%%%%%%%%%%%%%%%%%%%%%%%%%%%%%%%%%%%%%%%%%%%%%%%%
% Now the actual document:

\begin{document}
\fontfamily{ppl} \selectfont
% Name with horizontal rule
\bigname{\name}
\begin{center}
  \web
\end{center}
\vspace{-8pt} \rule{\textwidth}{1pt}


\begin{ressectionlist}{SKILLS}

  \begin{tabular*}{\textwidth}{@{\extracolsep{\fill} } l l l }
    Requirements Analysis & Software Development & Coding \& Testing\\
        Documentation & Debugging \& Troubleshooting & Performance Optimization \\
  \end{tabular*}
  \newline
\end{ressectionlist}


\begin{ressection}{TECHNOLOGY}

  \resitem{\textbf{Programming:} C/C++(proficient), OpenGL(proficient), GLSL(proficient), Rust(proficient), CUDA(proficient), OpenCL(proficient), MPI(beginner), Python(intermediate), Qt(proficient), Git(proficient), Android(beginner), Vulkan(beginner)}

  \resitem{\textbf{Tools \& Frameworks:} docker(intermediate), GIMP(intermediate), Photoshop(beginner), Matlab(beginner), \LaTeX{}(intermediate), Maya(beginner)}

\end{ressection}


\begin{ressection}{EXPERIENCE}
  \begin{ressubsec3}{Freelance Software Developer}{Self-Employed}{Dec 2015 - Present}
    \begin{blockquote}
      Core Services: Software Development, Performance Optimization, Porting
  \end{blockquote}
  \end{ressubsec3}

  \begin{ressubsec3}{Sr. Software Engineer}{ArrayFire}{Jun 2013 - Nov 2015}
    \begin{blockquote}
      Development and maintenance of ArrayFire library. Worked on Porting \& performance optimization projects involving CUDA/OpenCL/ArrayFire/OpenGL technologies.
    \end{blockquote}
  \end{ressubsec3}

  \begin{ressubsec3}{Graphics Programmer}{Visualization Department, Texas A\&M University}{Sep 2012 - Apr 2013}

    \begin{blockquote}
      Worked on an image processing application to handle foreground-background separation.
    \end{blockquote}

  \end{ressubsec3}

  \begin{ressubsec3}{Rendering Programmer}{USC Institute of Creative Technologies}{May 2012 - Jul 2012}

    \begin{blockquote}
      Worked on the analysis and extraction of subsurface-scattering parameters for an object.This work is published in IEEE Computer Graphics and Applications special issue.
    \end{blockquote}

  \end{ressubsec3}

  \begin{ressubsec3}{Help Desk Assistant}{Open Access Labs, Texas A\&M University}{Sep 2011 - Apr 2012}
  \end{ressubsec3}

  \begin{ressubsec3}{Assistant Systems Engineer}{Tata Consultancy Services Limited}{Jul 2008 - Jun 2010}

    \begin{blockquote}
      Provided production support for the web applications deployed on WebSphere Application Server. Automated app deployment on servers, drastically reducing the costs associated.
    \end{blockquote}

  \end{ressubsec3}

\end{ressection}


\begin{ressection}{EDUCATION}

  \begin{ressubsec3}{M.S. in Computer Science}{Texas A\&M University, College Station}{May 2013}
  \end{ressubsec3}

  \begin{ressubsec3}{B. Tech. in Computer Science and Engineering}{NIT-Durgapur, India}{May 2008}
  \end{ressubsec3}

\end{ressection}

\begin{ressection}{PUBLICATIONS}

  \resitem{Pradeep Garigipati, Ergun Akleman: Duotone Surfaces. International Symposium on Computational Aesthetics in Graphics, Visualization, and Imaging, 2012. Annecy, France. 99-106.}

  \resitem{Yufeng Zhu, Pradeep Garigipati, Pieter Peers, Paul Debevec and Abhijeet Ghosh. Estimating Diffusion Parameters from Polarized Spherical Gradient Illumination. IEEE Computer Graphics and Applications, 2013.}

\end{ressection}

\end{document}
